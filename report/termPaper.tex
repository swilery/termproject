%-----------------------------------------------------------------------------
%
%               Template for sigplanconf LaTeX Class
%
% Name:         sigplanconf-template.tex
%
% Purpose:      A template for sigplanconf.cls, which is a LaTeX 2e class
%               file for SIGPLAN conference proceedings.
%
% Guide:        Refer to "Author's Guide to the ACM SIGPLAN Class,"
%               sigplanconf-guide.pdf
%
% Author:       Paul C. Anagnostopoulos
%               Windfall Software
%               978 371-2316
%               paul@windfall.com
%
% Created:      15 February 2005
%
%-----------------------------------------------------------------------------


\documentclass{sigplanconf}

% The following \documentclass options may be useful:

% preprint      Remove this option only once the paper is in final form.
% 10pt          To set in 10-point type instead of 9-point.
% 11pt          To set in 11-point type instead of 9-point.
% authoryear    To obtain author/year citation style instead of numeric.

\usepackage{amsmath}


\begin{document}

\special{papersize=8.5in,11in}
\setlength{\pdfpageheight}{\paperheight}
\setlength{\pdfpagewidth}{\paperwidth}

\conferenceinfo{CONF 'yy}{Month d--d, 20yy, City, ST, Country}
\copyrightyear{20yy}
\copyrightdata{978-1-nnnn-nnnn-n/yy/mm}
\doi{nnnnnnn.nnnnnnn}

% Uncomment one of the following two, if you are not going for the
% traditional copyright transfer agreement.

%\exclusivelicense                % ACM gets exclusive license to publish,
                                  % you retain copyright

%\permissiontopublish             % ACM gets nonexclusive license to publish
                                  % (paid open-access papers,
                                  % short abstracts)

\titlebanner{banner above paper title}        % These are ignored unless
\preprintfooter{short description of paper}   % 'preprint' option specified.

\title{Elizabeth Scott Explained}
\subtitle{Parsing from Earley Recognisers}

\authorinfo{Zoe Wheeler}
           {University of Texas at Austin}
           {zoe.donnellon.wheeler@gmail.com}
\authorinfo{Walter Xia}
           {University of Texas at Austin}
           {swilery@utexas.edu}

\maketitle

\begin{abstract}
Earley's Algorithm is able to recognize general context-free grammars in $O(n^3)$, where $n$ is the size of the string to be recognized. However, there are times in which we want more than just a yes or no answer. There are times in which we want an actual parse tree, and for ambiguous grammars, there are times in which we want all possible parse trees. Fortunately, there is a paper by Dr. Elizabeth Scott, \cite{scott}, that presents a technique to produce a data structure known as a Shared Packed Parse Forest(SPPF), able to represent even an infinite number of parse trees. Unfortunately this paper is poorly written, making it very difficult to understand. Our paper is a re-explanation of Scott's techniques. It is agreed by many that Earley's Algorithm is also difficult to understand. Fortunately, there exists a data structure due to Dr. Gianfranco Bilardi and Dr. Keshav Pingali, \cite{bilardi-pingali}, known as Grammar Flow Graphs(GFGs) that significantly ease the understanding of the algorithm by reformulating parsing problems as path problems in a graph. Our technique will use GFGs.
\end{abstract}

\category{F.7.2}{Semantics and Reasoning}{Program Reasoning--Parsing}

% general terms are not compulsory anymore,
% you may leave them out
\terms
Context-Free Languages, Cubic Generalized Parsing, Earley Parsing 

\keywords
Earley Sets, Grammar Flow Graphs, Non-Deterministic Finite Automaton, Shared Packed Parse Forest

\section{Introduction}

The text of the paper begins here.

Lots of text.

More text.

Lots of text.

More text.


Lots of text.

More text.

Lots of text.

More text.


Lots of text.

More text.

Lots of text.

More text.

Lots of text.

More text.

Lots of text.

More text.

Lots of text.

More text.

Lots of text.

More text.

Lots of text.

More text.

Lots of text.

More text.

Lots of text.

More text.

Lots of text.

More text.


Lots of text.

More text.

Lots of text.

More text.


Lots of text.

More text.

Lots of text.

More text.

Lots of text.

More text.

Lots of text.

More text.

Lots of text.

More text.

Lots of text.

More text.

Lots of text.

More text.

Lots of text.

More text.


Lots of text.

More text.

Lots of text.

More text.




Lots of text.

More text.

Lots of text.

More text.

Lots of text.


Lots of text.

More text.

Lots of text.

More text.

Lots of text.

More text.

Lots of text.

More text.

Lots of text.

More text.

Lots of text.

More text.

Lots of text.

More text.

Lots of text.

More text.

Lots of text.

More text.

Lots of text.

More text.

Lots of text.

More text.

Lots of text.

More text.

Lots of text.

\appendix
\section{Appendix Title}

This is the text of the appendix, if you need one.

\acks

Acknowledgments, if needed.

% We recommend abbrvnat bibliography style.

\bibliographystyle{abbrvnat}

% The bibliography should be embedded for final submission.

\begin{thebibliography}{}
\softraggedright

\bibitem[Bilardi and Pingali(2012)]{bilardi-pingali}
Gianfranco Bilardi, and Keshav Pingali. Parsing with Pictures. UTCS Tech Reports, 2012. This is a full TECHREPORT entry.

\bibitem[Scott(2008)]{scott}
Elizabeth Scott. SPPF-Style Parsing From Earley Recognisers. \textit{Electronic Notes in Theoretical Computer Science}, 203(53-67), 2008. This is a full ARTICLE entry.

\end{thebibliography}


\end{document}

%                       Revision History
%                       -------- -------
%  Date         Person  Ver.    Change
%  ----         ------  ----    ------

%  2013.06.29   TU      0.1--4  comments on permission/copyright notices

